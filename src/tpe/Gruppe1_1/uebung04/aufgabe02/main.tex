\documentclass[12pt]{scrartcl}

\usepackage{listings}

% the following is needed for syntax highlighting
\usepackage{color}

\definecolor{dkgreen}{rgb}{0,0.6,0}
\definecolor{gray}{rgb}{0.5,0.5,0.5}
\definecolor{mauve}{rgb}{0.58,0,0.82}

\lstset{ %
  language=Java,                  % the language of the code
  basicstyle=\footnotesize,       % the size of the fonts that are used for the code
  numbers=left,                   % where to put the line-numbers
  numberstyle=\tiny\color{gray},  % the style that is used for the line-numbers
  stepnumber=1,                   % the step between two line-numbers. If it's 1, each line 
                                  % will be numbered
  numbersep=5pt,                  % how far the line-numbers are from the code
  backgroundcolor=\color{white},  % choose the background color. You must add \usepackage{color}
  showspaces=false,               % show spaces adding particular underscores
  showstringspaces=false,         % underline spaces within strings
  showtabs=false,                 % show tabs within strings adding particular underscores
  frame=single,                   % adds a frame around the code
  rulecolor=\color{black},        % if not set, the frame-color may be changed on line-breaks within not-black text (e.g. commens (green here))
  tabsize=4,                      % sets default tabsize to 2 spaces
  captionpos=b,                   % sets the caption-position to bottom
  breaklines=true,                % sets automatic line breaking
  breakatwhitespace=false,        % sets if automatic breaks should only happen at whitespace
  title=\lstname,                 % show the filename of files included with \lstinputlisting;
                                  % also try caption instead of title
  keywordstyle=\color{blue},          % keyword style
  commentstyle=\color{dkgreen},       % comment style
  stringstyle=\color{mauve},         % string literal style
  escapeinside={\%*}{*)},            % if you want to add a comment within your code
  morekeywords={*,...}               % if you want to add more keywords to the set
}


\begin{document}
\section*{Aufgabe 2}
In Java kommunizieren die Threads ueber den gemeinsamen genutzen Speicher (heap).
Java unterstuetzt die Synchronisierung von Threads mittels Monitoren, ueber Ereignisse 
oder durch das warten auf das Ende eines anderen Threads (join).
\\
Im Falle von Monitoren werden Threads durch bestimmte Abschnitte Synchronisiert. Ein
Monitor erlaubt dabei jeweils nur einem Thread den Zugriff auf die synchronisierte Stelle. 
Eine solche gesperrte Stelle kann man innerhalb einer Methode mit

\begin{lstlisting}{synchronize in Methode}
synchronized(this) { }
// oder fuer ein bestimmtes Objekt
synchronized(Objekt) { }
\end{lstlisting}

erzeugen. Man kann auch ganze Methoden synchronisieren

\begin{lstlisting}{synchronize auf Methode}
synchronized void m () { ... }
\end{lstlisting}

Dadurch hat an dieser Stelle immer nur ein Thread Zugriff, jeder weitere Thread der nun an 
diese Stelle kommt, traegt sich in eine Warteliste ein und muss nun warten bis der Thread, der
momentan im Monitor ist, diesen verlassen hat. Somit kann man kritische Stellen in einem Programm
kapseln und somit sicher gehen, das auch bei vielen Threads keine Probleme entstehen.
\\
Ein Thread kann die Sperre des Monitors temporaer zuruecksetzen indem er \textbf{\textit{wait}} 
benutzt. Ein Thread kann zusaetzlich einen (beliebigen) anderen Thread per \textbf{\textit{notify}} 
aus der Warteliste entfernen und die temporaer aufgehobenen Sperren wieder herstellen. Das gleiche 
passiert mit \textbf{\textit{notifyAll}} ausser das dort alle Threads aus der Warteliste entfernt werden. 
\textbf{\textit{notify}}, \textbf{\textit{notifyAll}} und \textbf{\textit{wait}} gelten nur fuer 
gesperrte Objekte. Zusaetzlich sollte \textbf{\textit{wait}} immer in einer Schleife aufgerufen werden 
damit auch immer ueberprueft werden kann ob ein gewollter Zustand eingetreten ist.
\\
Eine weitere Methode der Synchronisation ist \textbf{\textit{join}}. Mit dieser Methode wartet ein Thread 
bis der Thread auf den Join aufgerufen wurde fertig ist, optional kann man hier auch noch eine Zeit angeben 
die gewartet werden soll bevor automatisch weitergemacht wird.
\\
Sollte nicht synchronisiert werden, so koennen verschiedenste Probleme eintreten. Beispielsweise koennte 
ein Thread etwas auf eine int variable schreiben wenn diese kleiner als 50 ist (und dabei z.B. 50 addieren), 
sollte zu diesem Zeitpunkt ein anderer Thread gerade dieselbe abfrage gemacht haben und zu diesem Zeitpunkt war 
auch bei ihm der Wert kleiner als 50 koennte sich der Wert nun schon auf z.B. 90 abgeaendert haben. Der zweite 
Thread addiert nun trotzdem 50 hinzu und somit kommt es zu ungewollten Zustaenden.
\end{document}
